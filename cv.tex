%% Marcelo Armengot - CV in LaTeX
\documentclass[10pt,a4paper]{article}
\usepackage[utf8]{inputenc}
\usepackage[english]{babel}
\usepackage{geometry}
\geometry{margin=1in}
\usepackage{parskip}
\usepackage{enumitem}
\usepackage{hyperref}
\usepackage{xcolor}
\usepackage{graphicx}
\usepackage{utopia}


% Custom commands
\newcommand{\sect}[1]{\vspace{6pt}\noindent\textbf{\large #1}\vspace{2pt}\hrule\vspace{6pt}}
\newcommand{\subsect}[1]{\vspace{4pt}\textbf{#1}}

% \renewcommand{\familydefault}{\sfdefault}

\begin{document}

\pagestyle{empty}

\begin{center}
\begin{minipage}[c]{0.25\textwidth}
    \includegraphics[width=\linewidth]{profile.ps}
\end{minipage}
\hfill
\begin{minipage}[c]{0.7\textwidth}
    {\LARGE \textbf{Marcelo Armengot}} \\[4pt]
    \textit{\textbf{Computer Vision} senior dev} \\[4pt]
    Madrid, Spain \\[4pt]
    (Profile picture from Dreambooth Stable Diffusion) \\
    \begin{itemize}[leftmargin=*,label={}]
      \item \href{https://github.com/armengot}{github.com/armengot}
      \item \href{https://orcid.org/0000-0001-8101-9413}{orcid.org/0000-0001-8101-9413}
    \end{itemize}
\end{minipage}
\end{center}

~\\

\sect{Education}
\begin{itemize}[leftmargin=*]
  \item \subsect{PhD in Astrophysics}\\
    \textit{Universidad Complutense de Madrid} (2013 – 2016) \\
    Ultraviolet image processing for interstellar medium studies.

  \item \subsect{MSc in Computer Science and Computational Mathematics}\\
    \textit{Universitat de València} (2004 – 2006) \\
    Facial recognition using PCA/LDA and Common Vector methods.

  \item \subsect{Bachelor's in Computer Science (5-year)}\\
    \textit{Universitat de València} (1995 - 2000) \\
    Final project: Perl client/server app for GNU/Linux and Win32.
\end{itemize}

\sect{Publications}
\begin{itemize}[leftmargin=*]
    \item \textbf{Signatures of diffuse interstellar gas in the Galaxy Evolution Explorer all-sky survey} \\
    \textit{Marcelo Armengot and Ana Inés Gómez de Castro} \textbf{Astronomy and Astrophysics}, 2019.
    
    \item \textbf{MOSAIX: a tool to build large mosaics from GALEX images} \\
    \textit{Marcelo Armengot, Néstor Sánchez, Javier López-Santiago and Ana Inés Gómez de Castro} \textbf{Astrophysics and Space Science}, 2014.
\end{itemize}

\sect{Conferences}
\begin{itemize}[leftmargin=*]
    \item \textbf{Distribution of ISM features as per the GALEX all-sky survey} (\textit{Marcelo Armengot and Ana Inés Gómez de Castro}) \textit{Ultraviolet Sky Surveys “The needs and the means”}, 2017, Tel Aviv, Israel.

    \item \textbf{The properties of diffuse interstellar dust clouds as determined from GALEX and infrared (IRAS, Herschel) observations} (\textit{Marcelo Armengot, Ana Inés Gómez de Castro, Javier López-Santiago and Néstor Sánchez}) \textit{Highlights on Spanish Astrophysics IX}, XII Scientific Meeting of the Spanish Astronomical Society, 2016, Bilbao, Spain.

    \item \textbf{UV image processing to detect diffuse clouds} (\textit{Marcelo Armengot, Ana Inés Gómez de Castro, Javier López-Santiago and Néstor Sánchez}) \textit{XI Scientific Meeting of the Spanish Astronomical Society}, 2014, Teruel, Spain.

    \item \textbf{A GALEX based survey of the Taurus star forming region} (\textit{Ana Inés Gómez De Castro, J. López Santiago, F. López-Martínez, N. Sánchez, P. Sestito, M. Armengot, E. de Castro, M. Cornide and J. Yañez-Gestoso}) \textit{Protostars \& Planets VI Workshop}, 2013, Heidelberg, Germany.

    \item \textbf{Adaptive kernel ridge regression for image denoising} (\textit{Marcelo Armengot, Valero Laparra, Luis Gómez, Jesús Malo and Gustavo Camps-Valls}) \textit{IEEE Workshop on Machine Learning for Signal Processing (MLSP)}, 2010, Kittilä, Finland.

    \item \textbf{Experiments about the Generalization Ability of Common Vector based methods for Face Recognition} (\textit{Marcelo Armengot, Francesc Ferri and Wladimiro Diaz}) \textit{7th International Workshop of Pattern Recognition in Information Systems (PRIS)}, in conjunction with ICEIS 2007, Funchal, Portugal.
\end{itemize}


\newpage

\sect{Professional Experience}
\begin{itemize}[leftmargin=*]
  
\item \subsect{Senior Software Engineer — Indra} (\textit{2023 – Present})
\begin{itemize}[leftmargin=*,label={}]
\item Real-time C++ (CUDA) Computer Vision multithreading subsystem.
\end{itemize}
  
\item \subsect{Computer Vision Tech Lead — Izertis} (\textit{2022 – 2023})
\begin{itemize}[leftmargin=*,label={}]
\item Led and developed Computer Vision projects.
\end{itemize}

\item \subsect{Software Developer — Agerpix Technologies} (\textit{2020 - 2022})
\begin{itemize}[leftmargin=*,label={}]
    \item Image Processing and Machine Learning.
    \item Python, OpenCV, C++, R, QGIS, PostgreSQL, Shell scripting.
\end{itemize}

\item \subsect{Software Developer — Xeridia} (\textit{2019 –- 2019})
\begin{itemize}[leftmargin=*,label={}]
    \item C/C++/Pro*C Oracle on AIX for online banking.
\end{itemize}

\item \subsect{Computer System Manager — Atlantida} (\textit{2017 – 2019})
\begin{itemize}[leftmargin=*,label={}]
\item Remote support for multimedia servers onboard Renfe long-distance trains.
\end{itemize}

\item \subsect{Computer System Manager — Universidad Complutense de Madrid} (\textit{2012 – 2017})
\begin{itemize}[leftmargin=*,label={}]
    \item Computer support for WSO-UV project (servers, Perl, shell scripting).
    \item Research on ultraviolet image processing with a focus on the WSO-UV project.
\end{itemize}

\item \subsect{Research Staff — Universitat de València} (\textit{2009 – 2011})
\begin{itemize}[leftmargin=*,label={}]
    \item Kernel methods, regression, clustering, image denoising, neural networks.
    \item Image processing research at IPL (ipl.uv.es).
\end{itemize}

\item \subsect{Research Staff — Instituto Tecnológico de Informática (ITI)} (\textit{2007 – 2009})
\begin{itemize}[leftmargin=*,label={}]
\item C/C++, biometric libraries, optimization, OpenCV, surveillance projects.
\end{itemize}

\item \subsect{Research Scholar — Universitat de València} (\textit{2004 – 2007})
\begin{itemize}[leftmargin=*,label={}]
    \item Scheduling software (Delphi/C), C/GTK frontend for Graphviz dot graphs.
    \item E-learning platform, helpdesk, webmaster.
\end{itemize}

\end{itemize}

\sect{Skills}
\begin{itemize}[leftmargin=*]
    \item \textbf{Programming Languages:} C/C++, Python, Perl, Matlab, R, PHP.
    \item \textbf{Web \& Markup:} HTML/CSS, LaTeX.
    \item \textbf{Operating Systems:} Unix, Linux; shell scripting, awk.
    \item \textbf{Databases:} Oracle, PostgreSQL, MySQL.
    \item \textbf{Version Control \& IDEs:} Git, SVN, Eclipse.
    \item \textbf{Project Management Tools:} Jira, Confluence.
\end{itemize}

\sect{Personal Interests}
\begin{itemize}[leftmargin=*]
    \item Passionate about \textbf{cinema}, \textbf{creative writing} and \textbf{cooking}.
    \item Enjoy swimming, writing songs and playing the piano. My piano album on Jamendo: \href{https://www.jamendo.com/album/51633/piano-demo}{\textit{Piano Demo}}.
    \item Maintain a personal blog with thoughts, code and geeky stuff: \href{https://nerdinmadrid.tumblr.com/}{nerdinmadrid.tumblr.com}.
    \item Regular gamer – into both retro and modern titles.
\end{itemize}


\end{document}
